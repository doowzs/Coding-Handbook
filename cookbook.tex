\documentclass[11pt, a5paper, UTF8]{article}
\input{preamble}

\title{\textsf{ACM/OJ Handbook}}
\author{\textsf{Aunt *recovered version}}
\date{}
\begin{document}
\maketitle
\tableofcontents
\newpage

\section*{Head files}
\begin{lstlisting}
#include <set>
#include <map>
#include <cmath>
#include <queue>
#include <stack>
#include <vector>
#include <cstring>
#include <cstdio>
#include <cstring>
#include <iostream>
#include <algorithm>

using namespace std; //D A N G E R O U S ! ! !

const int MAXN = 100010;
const int INF = 0x3f3f3f3f;
\end{lstlisting}
	To set every element as inf, we just need to do memset(a,0x3f,sizeof(a)). \\

\section{STL Templates}

\subsection{Sort}
\begin{lstlisting}
bool compare(int a,int b) {
  return a < b; //1, 2, 3, ...
  //return a > b; //..., 3, 2, 1
}
sort(a, a+n, compare);
\end{lstlisting}
\subsection{QSort}
\begin{lstlisting}
int compare(const void *a,const void *b) {
  return *(int*)a > *(int*)b ? 1 : -1;   
}
qsort(a, 20, sizeof(int), compare);
//qsort(a, 20, sizeof(int), less<int>); or, greater<int>
\end{lstlisting}

\subsection{Vector}
\begin{lstlisting}
//declare
vector<string> v(5, "hello");
vector<string> v2(v.begin(), v.end());

//output
for(vector<string>::iterator it = v.begin();
      it < v.end(); ++it)
  cout<<*it<<endl;
//or
for(vector<string>::size_type i = 0; i < v.size(); ++i)
  cout<<v[i]<<endl;
  
//sort
sort(v.begin(), v.end());

//insert
v.insert(v.begin() + 3, 4, "hello, world");
//or push at back
v.push_back("1");

//erase, D A N G E R O U S ! ! !
vector<string>::iterator it =
    v.erase(v.begin() + 3, v.begin() + 6);
//or erase an element of specific key
for(vector<int>::iterator iter=veci.begin();
      iter!=veci.end(); ) {
  if( *iter == 3)
    iter = veci.erase(iter);
  else
    iter ++ ;
}

//others
void v.clear();
bool v.empty();
type v.front();
type v.back();
\end{lstlisting}
\ 
\subsection{Map}
\begin{lstlisting}
//declare
map <int, string> mapS;
map <int, double> mapD;

//insert
mapS->insert(map <int, string>::value_type(index, x));
//or, a function
void map_insert(map <int, string> *mapS,
                int index, string x) { 
  mapS->insert(map <int, string>::value_type(index, x));
}

//find and delete element 
map <int, string>::iterator iter = mapS.find(450); 
if (iter != mapS.end()) { 
  cout << "find the elememt" << endl; 
  cout << "It is:" << iter->second << endl; 
  map.erase(iter); 
} else { 
  cout << "404" << endl; 
}

//see element 
for (map <int, string>::iterator iter = mapS.begin();
         iter != mapS.end(); iter++) {
  cout << "| " << iter->first << " | " 
       << iter->second << " |" << endl;
}
\end{lstlisting}
\ 
\subsection{Useful Functions}
\begin{lstlisting}
min(int a, int b);
max(int a, int b);
swap(int a, int b);
reverse(a, a+n); //1, 2, 3 -> 3, 2, 1
next_permutation(a, a+n); //needs to be sorted
//the first key bigger than val (or end)
itr upper_bound(first, last, value, cmp); 
//the first key not smaller than val (or end)
itr lower_bound(first, last, value, cmp); 
\end{lstlisting}

\ 
\section{Prime Number}
\begin{lstlisting}
int prime[MAXN];  
bool notPrime[MAXN];

int sieve(int n){
  int p = 0;
  memset(notPrime, 0, sizeof(notPrime));
  notPrime[0] = notPrime[1] = true;
  for (int i = 2; i <= n; ++i){
    if(!notPrime[i]){
      prime[p++] = i;
      for(int j = i + i; j <= n; j += i)
        notPrime[j] = true;
    }
  }
  return p;
}
\end{lstlisting}
\ 
\section{Fast Power}
	$x$ is base, $n$ is power.
\begin{lstlisting}
long long powerMod(long long x, long long n) {
  long long res = 1;
  while (n > 0){
    if  (n & 1) //n is odd
      res = res * x;
    x = x * x;
    n >>= 1; //n = n / 2;
  }
  return res;
}
\end{lstlisting}

\section{GCD and LCM}
\subsection{Greatest Common Divisor}
\begin{lstlisting}
int gcd(int big, int small) { //non-recursive
  if (small > big) swap(big, small);
  int temp;
  while (small != 0){
    if (small > big) swap(big, small);
    temp = big % small;
    big = small;
    small = temp;
  }
  return(big);
}

int gcd(int a, int b) {  
 if (a < b)  
  swap(a, b);  
  
 if (b == 0)  
  return a;  
 else  
  return gcd(b, a%b);  
}

int ngcd(int *a, int n) {
 if (n == 1)  
  return *a;  
  
 return gcd(a[n-1], ngcd(a, n-1));  
}  
\end{lstlisting}
\subsection{Least Common Multiple}
\begin{lstlisting}
int lcm(int a, int b)  
{  
 return a*b/gcd(a, b);  
} 

int nlcm(int *a, int n) { 
 if (n == 1)  
  return *a;  
 else  
  return lcm(a[n-1], nlcm(a, n-1));  
}
\end{lstlisting}
\ 
\section{Permutation}
	$n$ numbers in the list, the first $k$ of them remains unchanged.
\begin{lstlisting}
void Pern(int list[], int k, int n) {
  if (k == n - 1) {
    for (int i = 0; i < n; i++) {
      printf("%d", list[i]);
    }
    printf("\n");
  }else {
    for (int i = k; i < n; i++) {
      swap(list[k], list[i]);
      Pern(list, k + 1, n);
      swap(list[k], list[i]);
    }
  }
}
\end{lstlisting}

\section{Binary Search}
	$left$ is the first element, $right$ is the \textbf{next element} of the end. $x$ is the target.
\begin{lstlisting}
int bsearch(int *A, int left, int right, int x) {
  int m;
  while (left < right){
    m = left + (right - left) / 2;
    if (A[m] >= x)  right = m;
    else left = m + 1;  
    //To find an upper_bound, switch to:
    //if (A[m] <= v) x = m+1; else y = m;   
  }
  return left;
}
\end{lstlisting}

\section{Union Find Set}
	The father of nodes are updated during search.
\begin{lstlisting}
int father[maxn];

void makeSet() {  
  for (int i = 0; i < maxn; i++)   
    father[i] = i;  
}

int findRoot(int x) {   //avoid recursion and RE
  int root = x;
  while (root != father[root]) {
    root = father[root];  
  }  
  while (x != root) {  
    int tmp = father[x];  
    father[x] = root;
    x = tmp;  
  }  
  return root;  
}  

void Union(int x, int y) {
  int a, b;  
  a = findRoot(x);  
  b = findRoot(y);  
  father[a] = b;
}  
\end{lstlisting}

\section{Minimum Spanning Tree}
\subsection{Kruskal} 
	Used for sparse graphs.
\begin{lstlisting}
void Kruskal() {  
  ans = 0;  
  for (int i = 0; i<len; i++) {  
    if (Find(edge[i].a) != Find(edge[i].b)) {  
      Union(edge[i].a, edge[i].b);  
      ans += edge[i].len;  
    }  
  }  
}  
\end{lstlisting}
\subsection{Prim}
	Used for dense graphs.
\begin{lstlisting}
struct node {  
  int v, len;  
  node(int v = 0, int len = 0) :v(v), len(len) {}  
  bool operator < (const node &a)const {  //define <
  return len> a.len;  
  }  
};

vector<node> G[maxn];
int vis[maxn];
int dis[maxn];

void init() {  
  for (int i = 0; i<maxn; i++) {  
  G[i].clear();  
  dis[i] = INF;  
  vis[i] = false;  
  }  
}  

int Prim(int s) {  
  priority_queue<node>Q; 
  int ans = 0;  
  Q.push(node(s,0));  //start node
  
  while (!Q.empty()) {   
  node now = Q.top();
  Q.pop();//the minimum distance
  
  int v = now.v;  
  if (vis[v]) continue;  //skip if visited
  vis[v] = true;  //mark visited
  
  ans += now.len;  
  //update the tree
  for (int i = 0; i<G[v].size(); i++) {
    int v2 = G[v][i].v;  
    int len = G[v][i].len;  
    if (!vis[v2] && dis[v2] > len) {   
    dis[v2] = len;
    //insert into the queue and sort
    Q.push(node(v2, dis[v2]));  
    }  
  }  
  }  
  return ans; 
}  
\end{lstlisting}

\section{Single-Source Shortest Path}
\subsection{Bellman-Ford}.
\begin{lstlisting}

\end{lstlisting}
\subsection{Dijkstra}
	Used for non-negative weights. Complexity $O(e\lg{n})$.
\begin{lstlisting}
struct node {  
  int v, len;  
  node(int v = 0, int len = 0) :v(v), len(len) {}  
  bool operator < (const node &a)const { 
    return len > a.len;  
  }  
};  

vector<node>G[maxn];  
bool vis[maxn];  
int dis[maxn];

void init() {  
  for (int i = 0; i<maxn; i++) {  
    G[i].clear();  
    vis[i] = false;  
    dis[i] = INF;  
  }  
}  
int dijkstra(int s, int e) {  
  priority_queue<node>Q;  
  Q.push(node(s, 0)); //insert and sort  
  dis[s] = 0;  
  while (!Q.empty()) {  
    node now = Q.top();   //get the minimum
    Q.pop();  
    int v = now.v;  
    if (vis[v]) continue;   //skip
    vis[v] = true;      //mark
    for (int i = 0; i<G[v].size(); i++) { //update
      int v2 = G[v][i].v;  
      int len = G[v][i].len;  
      if (!vis[v2] && dis[v2] > dis[v] + len) {  
        dis[v2] = dis[v] + len;  
        Q.push(node(v2, dis[v2]));  
      }  
    }  
  }  
  return dis[e];  
}  
\end{lstlisting}
\subsection{SFPA}
	Can deal with negative weights.
\begin{lstlisting}
vector<node> G[maxn];
bool inqueue[maxn];
int dist[maxn];

void init() {  
  for(int i = 0 ; i < maxn ; ++i){  
    G[i].clear();  
    dist[i] = INF;  
  }  
}

int SPFA(int s,int e) {  
  int v1,v2,weight;  
  queue<int> Q;  
  memset(inqueue,false,sizeof(inqueue));
  memset(cnt,0,sizeof(cnt)); //times of insertion 
  dist[s] = 0;  
  Q.push(s); //push the start
  inqueue[s] = true; //mark 
  while(!Q.empty()){  
    v1 = Q.front();  
    Q.pop();  
    inqueue[v1] = false; //cancel mark
    for(int i = 0 ; i < G[v1].size() ; ++i){ //traverse
      v2 = G[v1][i].vex;  
      weight = G[v1][i].weight;  
      if(dist[v2] > dist[v1] + weight){ //relaxation
        dist[v2] = dist[v1] + weight;  
        if(inqueue[v2] == false){  //insert again
          inqueue[v2] = true;  
          //cnt[v2]++;  //negative circle  
          //if(cnt[v2] > n) return -1;  
          Q.push(v2);  
        } } }  
  }  
  return dist[e];  
}
\end{lstlisting}
\subsection{Floyd-Warshall}
	The simplest and worst. Complexity $O(n^3)$.
\begin{lstlisting}
for (int i = 0; i < n; i++) {  
  for (int j = 0; j < n; j++)  
    scanf("%lf", &dis[i][j]);  
}  
for (int k = 0; k < n; k++) {  
  for (int i = 0; i < n; i++) {  
    for (int j = 0; j < n; j++) {  
      dis[i][j] = min(dis[i][j], dis[i][k]+dis[k][j]);  
    }  
  }
}
\end{lstlisting}

\section{Bipartite Graph}
\subsection{Judgement and Bipartite}
\begin{lstlisting}
int bipartite(int s) {  
  int u, v;  
  queue<int>Q;  
  color[s] = 1;  
  Q.push(s);  
  while (!Q.empty()) {  
    u = Q.front();  
    Q.pop();  
    for (int i = 0; i < G[u].size(); i++) {  
      v = G[u][i];  
      if (color[v] == 0) {  
        color[v] = -color[u];  
        Q.push(v);  
      }  
      else if (color[v] == color[u])  
        return 0;  
    }  
  }  
  return 1;  
}  
\end{lstlisting}
\subsection{Hungary Algorithm}
	Find maximum match.
\begin{lstlisting}
vector<int> G[maxn];  
bool inpath[maxn];
int match[maxn];

void init() {  
  memset(match, -1, sizeof(match));  
  for (int i = 0; i < maxn; ++i) {  
    G[i].clear();  
  }  
}  

bool findpath(int k) {  
  for (int i = 0; i < G[k].size(); ++i) {  
    int v = G[k][i];  
    if (!inpath[v]) {  
      inpath[v] = true;   
      //using recursion to find a way of matching
      if (match[v] == -1 || findpath(match[v])) {
        match[v] = k;
        return true;  
      }  
    }  
  }  
  return false;  
}  

void hungary() {  
  int cnt = 0;  
  for (int i = 1; i <= m; i++) {
    memset(inpath, false, sizeof(inpath)); 
    if (findpath(i)) cnt++;  
  }  
  cout << cnt << endl;  
}  
\end{lstlisting}

\section{Backpack Problems}
\subsection{0-1 Backpack}
\begin{lstlisting}
void bag01(int cost,int weight) {  
  for(i = v; i >= cost; --i)  
    if(dp[i]<dp[i-cost]+weight)  
      dp[i]=dp[i-cost]+weight; 
}  
\end{lstlisting}
\subsection{Complete Backpack}
\begin{lstlisting}
void complete(int cost,int weight) {  
  for(i=cost;i<=v;i++)  
    if(dp[i]<dp[i-cost]+weight)  
      dp[i]=dp[i-cost]+weight;  
}  
\end{lstlisting}
\subsection{Multiple Backpack}
\begin{lstlisting}
void multiply(int cost,int weight,int amount) {  
  if(cost*amount>=v)  
    complete(cost,weight);  
  else{  
    k=1;  
    while(k<amount){  
      bag01(k*cost,k*weight);  
      amount-=k;  
      k+=k;  
    }  
    bag01(cost*amount,weight*amount);  
  }  
}  
\end{lstlisting}
\ 
\section{Big Number}
	The first digit(0) stores how long the number is, the rest stores digits of the number. Use integer to avoid overflow!
\begin{lstlisting}
void add(int *a, int *b) { //a = a + b
  int length = max(a[0], b[0]);
  int sum[105] = {};
  for (int i = 1; i <= length; ++i) {
    sum[i] = a[i] + b[i];
  }
  for (int i = 1; i <= length; ++i) {
    sum[i + 1] += sum[i] / 10;
    sum[i] = sum[i] % 10;
  }
  if (sum[length + 1] > 0) sum[0] = length + 1;
  else sum[0] = length;

  for (int i = 0; i <= sum[0]; ++i) {
    a[i] = sum[i];
  }
}

void multiply(int *a, int *b) {
  //if (a[1] == 0 && a[2] == 0) getint();
  int length1 = a[0];
  int length2 = b[0];
  int sum[305] = {};
  int pos = 0;
  for (int i = 1; i <= length1; ++i) {
    pos = i;
    for (int j = 1; j <= length2; ++j) {
      sum[pos] += a[i] * b[j];
      pos++;
    }
  }
  //at this moment, pos is the new length.
  for (int i = 1; i <= pos+100; ++i) {
    sum[i + 1] += sum[i] / 10;
    sum[i] = sum[i] % 10;
  }
  pos = length1 + length2 + 100;
  while (sum[pos] == 0) pos--;
  sum[0] = pos;

  for (int i = 0; i <= sum[0]; ++i) {
    a[i] = sum[i];
  }
}

void printnum(int* a){
  for (int i = a[0]; i > 0; --i) {
    cout << (int) a[i];
  }
}
\end{lstlisting}
\ 
\section{LIS and LCS}
\subsection{Longest Increasing Subsequence}
\begin{lstlisting}
void solve() {
  for(int i = 0; i < n; ++i){  
    dp[i] = 1;  
    for(int j = 0; j < i; ++j){  
      if(a[j] < a[i]){  
        dp[i] = max(dp[i], dp[j] + 1);  
      }
    }
  }
}  
\end{lstlisting}
\ 
\subsection{Longest Common Subsequence}
\begin{lstlisting}
void solve() {  
  for (int i = 0; i < n; ++i) {  
    for (int j = 0; j < m; ++j) {  
      if (s1[i] == s2[j]) {  
        dp[i + 1][j + 1] = dp[i][j] + 1;  
      } else {  
        dp[i + 1][j + 1] = max(dp[i][j + 1],
                               dp[i + 1][j]);  
      }
    } 
  }
}  
\end{lstlisting}
\ 
\section{Geometry}
\subsection{Base}
\begin{lstlisting}
struct node {  
  double x; 
  double y; 
};  

typedef node Vector;

Vector operator + (Vector A, Vector B) {
  return Vector(A.x + B.x, A.y + B.y); 
}  
Vector operator - (Point A, Point B) {
  return Vector(A.x - B.y, A.y - B.y);
}  
Vector operator * (Vector A, double p) {
  return Vector(A.x*p, A.y*p);
}  
Vector operator / (Vector A, double p) {
  return Vector(A.x / p, A.y*p); 
}  

double Dot(Vector A, Vector B) {
  return A.x*B.x + A.y*B.y;
  } 
double Length(Vector A) {
  return sqrt(Dot(A, A));
} 
double Angle(Vector A, Vector B) {
  return acos(Dot(A, B) / Length(A) / Length(B));
}

double Cross(Vector A, Vector B) {
  return A.x*B.y - A.y*B.x;  
}  

Vector Rotate(Vector A, double rad) {  
  return Vector(A.x*cos(rad) - A.y*sin(rad),
          A.x*sin(rad) + A.y*cos(rad));  
}  

Point getLineIntersection(Point P, Vector v,
              Point Q, Vector w) {   
  Vector u = P - Q;   
  double t = Cross(w, u) / Cross(v, w);  
  return P + v*t;
}  
\end{lstlisting}
\ 
\subsection{Size of a polygon}
\begin{lstlisting}
node G[maxn];  
int n;

double getSize() {  
  double sum = 0;  
  G[n].x = G[0].x;  
  G[n].y = G[0].y;  
  for (int i = 0; i < n; i++) {   
      sum += Cross(G[i], G[i + 1] % n);  
  }
  //otherwise:
  sum = sum / 2.0;  
  return sum;
}
\end{lstlisting}
\subsection{Judge two line intersects}
\begin{lstlisting}
double Cross_Prouct(node A,node B,node C) {//BA Cross CA
    return (B.x-A.x)*(C.y-A.y)-(B.y-A.y)*(C.x-A.x);      
}      
bool Intersect(node A,node B,node C,node D) { 
  if(min(A.x,B.x)<=max(C.x,D.x)&&     
      min(C.x,D.x)<=max(A.x,B.x)&&      
      min(A.y,B.y)<=max(C.y,D.y)&&      
      min(C.y,D.y)<=max(A.y,B.y)&&      
      Cross_Prouct(A,B,C)*Cross_Prouct(A,B,D)<0&&      
      Cross_Prouct(C,D,A)*Cross_Prouct(C,D,B)<0)      
    return true;      
  else return false;      
}    
\end{lstlisting}

\section{KMP}
\begin{lstlisting}
void getnext(char str[maxn], int nextt[maxn]) {
  int j = 0, k = -1;
  nextt[0] = -1;
  while (j < m) {
    if (k == -1 || str[j] == str[k]) {
      j++;
      k++;
      nextt[j] = k;
    }
    else
      k = nextt[k];
  }
}

void kmp(int a[maxn], int b[maxn]) {  
  int nextt[maxm];  
  int i = 0, j = 0;  
  getnext(b, nextt);  
  while (i < n) {  
    if (j == -1 || a[i] == b[j]) {  
      j++;  
      i++;  
    }  
    else {
      j = nextt[j];  
    }  
    if (j == m) {  
      printf("%d\n", i - m + 1);  
      return;  
    }  
  }  
  printf("-1\n");
}  
\end{lstlisting}

\section{Tree}
\subsection{Segment Tree (node add)}
\begin{lstlisting}
struct node
{
  int left, right;
  int max, sum;
};

node tree[maxn << 2];
int a[maxn];
int n;
int k = 1;
int p, q;
string str;

void build(int m, int l, int r) {
  tree[m].left = l;
  tree[m].right = r;
  if (l == r){
    tree[m].max = a[l];
    tree[m].sum = a[l];
    return;
  }
  int mid = (l + r) >> 1;
  build(m << 1, l, mid);
  build(m << 1 | 1, mid + 1, r);
  tree[m].max = max(tree[m << 1].max, 
                    tree[m << 1 | 1].max);
  tree[m].sum = tree[m << 1].sum 
                 + tree[m << 1 | 1].sum;
}

//a is position, val is value
void update(int m, int a, int val) {
  if (tree[m].left == a && tree[m].right == a){
    tree[m].max += val;
    tree[m].sum += val;
    return;
  }
  int mid = (tree[m].left + tree[m].right) >> 1;
  if (a <= mid){
    update(m << 1, a, val);
  }
  else{
    update(m << 1 | 1, a, val);
  }
  tree[m].max = max(tree[m << 1].max, 
                    tree[m << 1 | 1].max);
  tree[m].sum = tree[m << 1].sum 
                 + tree[m << 1 | 1].sum;
}

int querySum(int m, int l, int r)
{
  if (l == tree[m].left && r == tree[m].right){
    return tree[m].sum;
  }
  int mid = (tree[m].left + tree[m].right) >> 1;
  if (r <= mid){
    return querySum(m << 1, l, r);
  }
  else if (l > mid){
    return querySum(m << 1 | 1, l, r);
  }
  return querySum(m << 1, l, mid) 
          + querySum(m << 1 | 1, mid + 1, r);
}

int queryMax(int m, int l, int r)
{
  if (l == tree[m].left && r == tree[m].right){
    return tree[m].max;
  }
  int mid = (tree[m].left + tree[m].right) >> 1;
  if (r <= mid){
    return queryMax(m << 1, l, r);
  }
  else if (l > mid){
    return queryMax(m << 1 | 1, l, r);
  }
  return max(queryMax(m << 1, l, mid), 
             queryMax(m << 1 | 1, mid + 1, r));
} 

build(1,1,n);  
update(1,a,b);  
query(1,a,b);  
\end{lstlisting}
\ 
\subsection{Segment Tree (section add)}
\begin{lstlisting}
int t,n,q;  
ll anssum;  

struct node{  
  ll l,r;  
  ll addv,sum;  
}tree[maxn<<2];  

void maintain(int id) {  
  if(tree[id].l >= tree[id].r)  
    return ;  
  tree[id].sum = tree[id<<1].sum + tree[id<<1|1].sum;  
}  

void pushdown(int id) {  
  if(tree[id].l >= tree[id].r)  
    return ;  
  if(tree[id].addv){  
    int tmp = tree[id].addv;  
    tree[id<<1].addv += tmp;  
    tree[id<<1|1].addv += tmp;  
    tree[id<<1].sum += 
        (tree[id<<1].r - tree[id<<1].l + 1)*tmp;  
    tree[id<<1|1].sum += 
        (tree[id<<1|1].r - tree[id<<1|1].l + 1)*tmp;  
    tree[id].addv = 0;  
  }  
}  

void build(int id,ll l,ll r) {  
  tree[id].l = l;  
  tree[id].r = r;  
  tree[id].addv = 0;  
  tree[id].sum = 0;  
  if(l==r)  {  
    tree[id].sum = 0;  
    return ;  
  }  
  ll mid = (l+r)>>1;  
  build(id<<1,l,mid);  
  build(id<<1|1,mid+1,r);  
  maintain(id);  
}  

void updateAdd(int id,ll l,ll r,ll val) {  
  if(tree[id].l >= l && tree[id].r <= r)  
  {  
    tree[id].addv += val;  
    tree[id].sum += (tree[id].r - tree[id].l+1)*val;  
    return ;  
  }  
  pushdown(id);  
  ll mid = (tree[id].l+tree[id].r)>>1;  
  if(l <= mid)  
    updateAdd(id<<1,l,r,val);  
  if(mid < r)  
    updateAdd(id<<1|1,l,r,val);  
  maintain(id);  
}  

void query(int id,ll l,ll r) {  
  if(tree[id].l >= l && tree[id].r <= r){  
    anssum += tree[id].sum;  
    return ;  
  }  
  pushdown(id);  
  ll mid = (tree[id].l + tree[id].r)>>1;  
  if(l <= mid)  
    query(id<<1,l,r);  
  if(mid < r)  
    query(id<<1|1,l,r);  
  maintain(id);  
}  

int main() {  
  scanf("%d",&t);  
  int kase = 0 ;  
  while(t--){  
    scanf("%d %d",&n,&q);  
    build(1,1,n);  
    int id;  
    ll x,y;  
    ll val;  
    printf("Case %d:\n",++kase);  
    while(q--){  
      scanf("%d",&id);  
      if(id==0){  
        scanf("%lld %lld %lld",&x,&y,&val);  
        updateAdd(1,x+1,y+1,val);  
      }  
      else{  
        scanf("%lld %lld",&x,&y);  
        anssum = 0;  
        query(1,x+1,y+1);  
        printf("%lld\n",anssum);  
      } 
    } 
  }  
  return 0;  
}  
\end{lstlisting}
\ 
\subsection{Binary Indexed Tree}
\begin{lstlisting}
int lowbit(const int t) {
  return t & (-t);
}

void insert(int t, int d) {
  while (t <= n){
    a[t] += d;
    t = t + lowbit(t);
  }
}

ll getSum(int t) {
  ll sum = 0;
  while (t > 0){
    sum += a[t];
    t = t - lowbit(t);
  }
  return sum;
}
\end{lstlisting}
\end{document}